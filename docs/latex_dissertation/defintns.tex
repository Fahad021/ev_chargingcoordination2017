%%
%% defintns.tex - User defined commands for edengths.tex
%%
%% Copyright (C) 2012 Mathew Topper <damm_horse@yahoo.co.uk>
%%
%% This file is part of the University of Edinburgh, Department of
%% Engineering LaTeX2e thesis template.
%% 
%% The University of Edinburgh, Department of Engineering LaTeX2e thesis
%% template is free software: you can redistribute it and/or modify
%% it under the terms of the GNU General Public License as published by
%% the Free Software Foundation, either version 3 of the License, or
%% (at your option) any later version.
%% 
%% The University of Edinburgh, Department of Engineering LaTeX2e thesis
%% template is distributed in the hope that it will be useful,
%% but WITHOUT ANY WARRANTY; without even the implied warranty of
%% MERCHANTABILITY or FITNESS FOR A PARTICULAR PURPOSE.  See the
%% GNU General Public License for more details.
%% 
%% You should have received a copy of the GNU General Public License
%% along with the University of Edinburgh, Department of Engineering
%% LaTeX2e thesis template.  If not, see <http://www.gnu.org/licenses/>.
%%
%%
%%   ABOUT
%%
%% This file contains the user defined commands for a Latex2e template which
%% corresponds to the regulations regarding layout of a thesis submitted within
%% the University of Edinburgh School of Engineering. It is not `official',
%% but conforms as best as possible to the regulation as detailed at:
%%
%%   http://www.scieng.ed.ac.uk/Postgraduate/PhD/settingoutyourthesis.htm
%%
%% Please feel free to alter the template to your own liking, but note that
%% the template is made available under the GNU GPL and must be similarly
%% licenced should you wish to release your modified template.
%%
%%
%%   CREDITS
%%
%% This template is an amalgamtion of an existing Edinburgh University,
%% Electrical Engineering PhD Thesis class file (jthesis-v1.cls) authored by
%% George S Taylor which was released under the GNU GPL.
%% Code is included from the dmathesis class Written by M. Imran
%% for a thesis according to the university of Durham regulation, which was
%% released without copyright. It also contains ideas (possibly code) from the
%% Princeton thesis class file (PrincetonThesis.cls), authored by Mike Nolta.
%% Mathew Topper, Eoghan Maguire and Bill Edwards forsaw the need to maintain a
%% more recent latex implementation of the thesis regulations and thus, this
%% project was born. It is hoped that the template will be maintained by the
%% Edinburgh Engineering PhD community once released.

%%%%%%%%%%%%%%%%%%%%%%%%%%%%%%%%%%%%%%%%%%%%%%%%%%%%%%%%%%%%%%%%%%%%%%%%%%
%%%%%%%%%%             Define your commands here              %%%%%%%%%%%%
%%%%%%%%%%%%%%%%%%%%%%%%%%%%%%%%%%%%%%%%%%%%%%%%%%%%%%%%%%%%%%%%%%%%%%%%%%

%% New commands can be written using
%%    \newcommand{command}[inputs]{definition}.
%% In the definition the inputs are accessed with #1, #2, etc.
%%
%% If you want to override an existing command use \renewcommand instead
%% of \newcommand. \newcommand with give an error if command is already
%% defined.

%% If you are concerned that your command might override a default
%% you can use \providecommand which will ignore the new command if
%% a command of that name already exists.

%%%%% Some Example Maths Definitions (only use in maths mode)

\newcommand{\pdif}[2]{\frac{\partial #1}{\partial #2}}
%% ie \pdif{x}{t} would give partial x over t.

\newcommand{\dpdif}[2]{\dfrac{\partial #1}{\partial #2}}
%% inline partial derivative ie for $\dpdif{x}{t}$.
%% (\dfrac needs amsmath package)

\newcommand{\Ddif}[2]{\frac{D #1}{D #2}}
%% Material derivative

\newcommand{\spdif}[2]{\frac{\partial^{2} #1}{\partial #2^{2}}}
%% second partial derivative.

\newcommand{\altspdif}[3]{\frac{\partial^{2} #1}{\partial #2 \partial #3}} 
% mixed second partial i.e. \altspdif{x}{z}{t} = d2x / dzdt

\newcommand{\ndif}[2]{\frac{\mathrm{d} \, #1}{\mathrm{d} #2}}
%% ordinary differential.

\newcommand{\sndif}[2]{\frac{\mathrm{d} \,^{2} #1}{\mathrm{d} #2^{2}}}
%% ordinary second differential.
