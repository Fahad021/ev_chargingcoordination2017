\chapter{Foundations and Literature Review}
\label{sec:found}

This chapter provides an introduction to the academic literature on the grid integration of electric vehicles. \Autoref{sec:impact} focusses on impact assessments of uncontrolled EV loads on various power network levels and outlines the potential benefit of controlled EV loads. Consequently, \Autoref{sec:deterlr} deals with previous approaches to optimise EV charging processes and compiles a summary of the different objectives and scopes of EV scheduling. Because reliability and robustness despite numerous uncertainties in scheduling are a major concern, \Autoref{sec:stochlr} regards related work that elaborated on scheduling methods considering forecast deviations. Finally, \Autoref{sec:barr} sketches out why EV owners might be hesitant to procure electric vehicles and participate in devised charging control. It deduces the significance of high cost savings and demand satisfaction rates in EV scheduling for an accelerated, sustainable, and grid-wise benign market uptake of electric vehicles.

\section{Impact and Potential of Electric Vehicles in Power Networks}
\label{sec:impact}

% GENERAL
The adoption of EVs will introduce new demand patterns into the power network, and assessment of their impact has received much attention in research over the past decade. Reference \cite{OConnell2014} notes that a majority of EV charging will take place on low voltage unbalanced networks and consequently the distribution feeder is particularly affected and inevitably requires remediation \cite{Ochoa2012, Lakervi2007}. It is estimated that uncontrolled EV loads can be accommodated in low-voltage distribution networks up to penetration rates between 20\% and 30\% without major technical problems \cite{Taylor2009, Quiros-Tortos2016, Richardson2010}. It is further shown that network strains are most severe during periods of simultaneous charging of a vast number of electric vehicles and peak residential demand. Two comprehensive literature reviews on impact studies are published in \cite{Papadopoulos2012, Dubey2015}. They categorise three main areas of concern summarised in \Autoref{tab:concern}: electricity generation adequacy, equipment strain, and power quality.

\begin{table}[]
	\centering
	\begin{tabular}{@{}lll@{}}
		\toprule
		\textbf{Generation Adequacy}    	& \textbf{Equipment Strain}            & \textbf{Power Quality}    \\ \midrule
		\tabitem security of supply     	& \tabitem transformer life and safety & \tabitem voltage drops    \\
		\tabitem carbon intensity       	& \tabitem cable life and safety       & \tabitem phase imbalances \\
		\tabitem power reserve availability & \tabitem battery life                & \tabitem power losses     \\
		\tabitem distributed generation 	&                                      & \tabitem harmonics        \\ \bottomrule
	\end{tabular}
	\caption{Major areas of concern regarding an increasing number of uncontrolled EV loads}
	\label{tab:concern}
\end{table}

% POWER QUALITY

Power quality comprises multiple dimensions. \cite{Boulanger2011,Wu2011,Pillai2010} demonstrate detrimental impacts of uncontrolled EV loads on voltage profiles in low-voltage residential distribution networks. \mbox{\cite{Richardson2010} conduct} an impact assessment of varying penetration levels of electric vehicles and show that already for modest penetration rates between 20\% and 40\%  voltage alongside line loadings can exceed statutory limits and would force the DNO to curtail its energy supply. Besides voltage issues, \cite{Clement-Nyns2010} point towards increased power losses in distribution networks. The results are confirmed in \cite{Paudyal2011} and \cite{Papadopoulos2012} found line losses to be as high as 10\% for high EV penetrations. \mbox{\cite{Putrus2009} add} phase load imbalances and subsequently voltage imbalances to the list of power quality concerns, while \cite{Koyanagi1999} analyses the development of power system harmonics. Stochastic analyses in \cite{Leou2013} verify the results of the mostly scenario-based studies.

% EQUIPMENT STRAIN
Alongside \cite{Richardson2010}, \cite{Papadopoulos2012} report frequent cable and transformer overloads and estimate that equipment stress is even more critical than voltage violations. 
\cite{Gong2012} study how uncontrolled EV charging negatively affects transformer insulation life. Likewise, \cite{Gomez2002, Shao2009, Warweg2011} analyse overloading and deterioration of transformer life due to total harmonic distortion (THD) of battery charger currents.

% GENERATION ADEQUACY
Although insufficient generation adequacy is a commonly stated concern, \cite{Webster1999, Hadley2009, Kintner-Meyer2007} conclude that despite growing EV loads existing and planned generation capacities suffice to guarantee the security of supply. Nonetheless, \cite{Dubey2015} note that uncontrolled EV loads may entail a significant reduction in power reserve margins. In \cite{Marwitz2016, Rassaei2014} the grid impact of EV loads against the backdrop of long-term planning regarding network reinforcement is analysed and underline the prohibitive costs of upgrading network equipment.

% CONCLUSION
Most papers conclude that technical difficulties can be overcome by controlled charging, which is illustrated with more or less rigorous charging strategies and assumptions. Consensus exists that existing distribution networks can accommodate substantial penetration levels of electric vehicles if the majority of charging is restricted to slow charging rates in off-peak periods \cite{Richardson2012a}. While \cite{Schuller2015a} quantify the load flexibility of electric vehicles and attest significant degrees of freedom, \cite{Quiros-Tortos2016} show that a control algorithm with control cycles of up to 10 minutes successfully mitigates power quality and equipment strain problems. Noteworthily, networks driven by high shares of renewable energy were shown to gain the most system cost reduction \cite{Shortt2012}. This suggests that the exploitation of load flexibilities by remote scheduling bears the potential to turn EV loads from a burden to a facilitator for the grid integration of renewable energy even beyond the additional demand the widespread adoption of EVs incurs.

\section{Deterministic Approaches to EV Load Scheduling}
\label{sec:deterlr}

Consequently, a vast amount of research was already conducted on the intelligent coordination of EV charging to harvest demand flexibility. The literature comprises a multitude of optimisation objectives, constraints, techniques, hierarchies, and scenarios. Comprehensive literature reviews in \cite{Mukherjee2015, Rajakaruna2015, Tan2016a} provide detailed insights into related studies examining conceivable applications deterministically; i.e.\ their work focusses on the theoretical potential and abstracts from operational uncertainties involved.

Besides the question whether discharge capabilities of electric vehicles are profitable, often the different control and communication architectures in which scheduling is embedded are discussed. Broadly central and decentral scheduling approaches are to be distinguished, whereby hybrid hierarchical control schemes are also conceivable. 

\textit{Central control} arranges charging processes by direct load control through a central scheduling instance and primarily focusses on reliable and safe network operation \cite{Vaya2012, Schuller2013}. Although new scheduling algorithms would integrate well with existing power system control, privacy concerns due to high requirements for information exchange as well as poor scalability for complex optimisation procedures could be prohibitive \cite{Li2005, OConnell2014}. 

Conversely, \textit{decentral control} retains the charging control of EV owners. Based on price discrimination with respect to time and location owners decide on when and to what objective to schedule, while pricing mirrors the technical status of the distribution network \cite{Rajakaruna2015}. Compared to central control it requires only minimal communication infrastructure mitigating data protection concerns and primarily aspires to fulfil EV owners' desires \cite{OConnell2014}. Furthermore, the division of the scheduling problem into subsets aids the problem scalability. Consequently, a group of advocates of decentral charging control exists \cite{He2012, Contreras-Ocana2016, Xing2015, Gan2013}.

While the use of standard functions to solve mixed-integer linear and convex non-linear problems dominate the literature \cite{Wu2012,Akhavan-Rezai2016}, dynamic programming \cite{Han2010, Rotering2011} and meta-heuristics such as particle-swarm optimisation \cite{Hai-Ying2011, Arias2017, Peppanen2014, Soares2013,Celli2012, Wen2016} have received growing attention to solve scheduling problems more quickly and efficiently. Correspondingly, \cite{Sassi2016} compare the complexity between exact and heuristic methods.

%\subsection{Technical Objectives}

The literature review in \cite{Rajakaruna2015} organises optimisation objectives into technical, environmental and economic objectives. It acknowledges, however, that a sharp distinction between the categories is not always feasible.

\textbf{Technical objectives} primarily target to maintain safe grid operation despite EV loads and intermittent renewable energy supply. In the distribution system, applications for peak shaving and load reduction prevail \cite{Mets2010, Turker2013, Wang2013a}. Thereby, \cite{Lopes2010, Mets2010} could increase the number of tolerable EVs from 10\% to 52\% by reducing peak load by 40\%. Further research evolves around the minimisation of losses and equipment strain \cite{Deilami2011, Sortomme2011} or even reactive power compensation \cite{Tan2016a}. \cite{Acha2012} achieved a loss reduction by 25\% for evaluated scenarios by coordinated charging. In addition to remedying thermal line loadings, \cite{Sundstrom2012, Connell2012} aimed and succeeded at stabilising voltage levels in residential areas. Similarly, in \cite{Richardson2012,Richardson2012a} a maximisation of the toal amount of energy that can be delivered in a distribution network while ensureing that network and voltage limits are never exceeded due to high levels of coincident charging. The results are confirmed by \cite{Richardson2010, Huang2012} which additionally seek to minimise phase imbalances \cite{Richardson2012}.

Alternative strategies for superordinate transmission systems apply as \cite{Vaya2012, Salah2012} noted a discrepancy between system state of the distribution system and transmission system. Studies on the transmission system level focus on the unit commitment problem regarding the carbon intensity and operation costs on the generation site \cite{Kiviluoma2011, Sioshansi2009}. 

The vehicle-to-grid (V2G) was laid out by \cite{Kempton1997} and entails an increased need for communication infrastructure \cite{Quinn2009}. Controlled EVs provide ancillary services such as spinning reserve and frequency regulation support \cite{Lopes2010, Tomic2007, Andersson2010}. V2G services are naturally linked to revenue generation from energy feed-back, paid per unit of energy, and capacity reserve provision, paid by capacity and time\cite{Kempton2005a}. \cite{Dallinger2011, Sortomme2011} show that negative regulation, namely charging in energy surplus periods, is most profitable while positive regulation would require additional investments. In \cite{Peterson2010} it is further demonstrated that capacity services rather contains the detrimental impact on battery lifetime.

\textbf{Economic objectives} eclipse the technical benefit of V2G and prioritise profits \cite{Sortomme2012}. Many also seek to exploit the load shifting potential by minimising the procurement cost of required energy on ordinary wholesale markets. Thereby, technical constraints are often only considered on a system level given implicitly through the price-building mechanism. While \cite{Guo2016} show the financial benefit to providers of V2G services, \cite{Cao2012} demonstrate how optimised charging pattern has great benefit in reducing cost and flatting the load curve if the peak and valley time periods are partitioned appropriately. Caution is however required as purely price-responsive charging according to uniform pricing based on the wholesale market may entail new peaks of EV loads \cite{Flath2013}.

\textbf{Environmental objectives} are usually intertwined with economic and technical objectives, and target to remedy the impact of additional volatile renewables, reduce fossil fuel dependency and reduce greenhouse gas emissions. The classic goal is to increase the utilisation share of renewables when charging by coupling renewable generation with EV loads \cite{Richardson2013, Vandael2011} and predominantly focus on wind power integration \cite{Pehnt2011, Short2006, Denholm2006}. In \cite{Druitt2012} it is derived that in a scenario where 30\% of electricity in the UK is supplied by wind power, one million electric vehicles could provide half the required balancing power. This order of magnitude is verified in \cite{Ekman2011}. Study \cite{Goransson2010} achieved emission reductions and efficiency increases in thermal generation due to less frequent start-ups and part load operation as intermittency is moderated by electric vehicles. However, regarding V2G applications \cite{Kempton2005, Kempton2006} note that EVs can only provide short-term storage of less than 2 hours, but may not attenuate daily or weekly generation variation.

In conclusion, potential applications of coordinating the charge rates of electric vehicles are diverse, but market-based and network-based optimisation are often disjunct. That is, a cost-minimising algorithm may disregard network constraints, or a peak-shaving algorithm neglects potential economic benefits. 

\newpage
\section{Stochastic Approaches to EV Load Scheduling}
\label{sec:stochlr}

In light of the prevailing uncertainties, it further needs to be acknowledged that charging coordination is no static offline problem but must incorporate possible forecast errors already while optimising \cite{Aghaei2016}. As compiled in \Autoref{tab:unclr}, several papers have presented diverse approaches to tackle uncertainties in recent years primarily addressing spot market prices, intermittent renewable electricity production, and EV user behaviour individually. Other than deterministic scheduling methods, economic cost minimisation objectives dominate, although they may be intertwined with environmental objectives, for instance by targeting maximum self-consumption of renewables \cite{Mehri2017, Jin2014, Bai2015}. Further, the literature offers a broad range of solution methods and problem formulations evolving around receding horizon control, stochastic and robust optimisation. 

\textbf{Robust optimisation} is predominantly applied to hedge against uncertain spot market prices. \cite{Soroudi2014} develop a scheduling approach based on a bi-level robust optimisation formulation that is immune to worst-case price prediction errors up to a controlled degree of risk aversion while considering nonlinear power flow. \cite{Korolko2015} criticise their negligence of physical battery limitations and the tractability of the proposed formulation. Instead, they present a cutting plane method to achieve robust solutions on an unregulated electricity market considering the nonlinear state-of-charge curve of batteries. Using a mixed-integer quadratic programming model, \cite{Bai2015} underline the benefit of aggregators participating in regulation services by reducing the total cost of thermal generators and containing the uncertainty of network operation costs.

\textbf{Stochastic optimisation problems} have been solved by information gap decision theory \cite{Zhao2015, Al-Awami2012}, Lyapunov optimisation \cite{Zhou2017}, Markov decision processes \cite{Zhang2014, Shi2011} as well as Monte Carlo methods \cite{Wu2017,Al-Awami2012}, scenario tree reduction \cite{Mehri2017}, and dynamic stochastic optimisation \cite{Liu2017}. To solve the EV scheduling problem by a Markov decision process in conjunction with a \mbox{Q-learning} algorithm, \cite{Shi2011} model price uncertainty via a Markov chain with unknown transition probabilities. \cite{Zhao2015} could formulate effective strategies to guarantee a predefined profit for risk-averse aggregators and demonstrate the trade-off between desired profits and preferred risk levels. Complementary, \cite{Zhou2017} point out that for their algorithm, the trade-off between cost and average fulfilment ratio of charging requests is at most linear.

\textbf{Receding optimisation horizons} are employed in \cite{Wang2017, Vaya2015, OConnell2014} to allow using available knowledge of preceding periods and updated more accurate forecasts to adapt optimised schedules. Using sequential quadratic programming, \cite{OConnell2014} could achieve an effective reduction in forecast error impacts ascribed to a rolling optimisation horizon to limit the period of agnostic decision-making. In a decentralised scheduling approach, \cite{Vaya2015} utilise the alternating direction method of multipliers (ADMM) to mitigate user mobility uncertainty and emphasise the benefit of parallel computation of local optimisation problems for scalability. Receding horizon control is further employed in conjunction with virtual load constraints in \cite{Wang2017}. They use model predictive control (MPC) to mitigate stochastic user behaviour modelled by kernel-based estimators.

Three frequently neglected issues detected motivate the scope of this dissertation. First, among the papers addressing various domains of uncertainty, few focus on the technical constraints of distribution networks \cite{Mehri2017, OConnell2014, Soroudi2014}. This contradicts the argument in \Autoref{sec:impact} that particularly low-voltage networks are sensible to additional EV loads, and therefore, deserves more attention. Second, stochastic and robust optimisation problems add substantially to the complexity of solution methods. The question arises whether plainer deterministic optimisation procedures would facilitate the adoption of remote scheduling while achieving equal levels of robustness if conservative parameter estimates are employed. A literature review on \mbox{state-of-the-art} decision making under uncertainty in energy systems is provided by \cite{Soroudi2013b}. Their intricacy repeatedly appears to fuel the phenomenon that models are suited to allow a particular optimisation method and not vice versa. Third, the common focus on individual aspects of uncertainty and less extensive model scopes further limit the comprehensive and realistic modelling of the charging problem and impact assessment. Furthermore, the cost and benefit of robustness are only partially evaluated.

% Please add the following required packages to your document preamble:
% \usepackage{booktabs}
% \usepackage{multirow}
\begin{landscape}
	\begingroup
	%\fontsize{9pt}{10pt}\selectfont
	\small
	\singlespacing
	%\renewcommand*{\baselinestretch}{1.0}
	\begin{longtable}{@{}p{0.5cm} >{\raggedright}p{4cm} >{\raggedright}p{3cm} >{\raggedright}p{3cm}  p{0.2cm}  p{0.2cm}  p{0.2cm}  p{0.2cm}  p{0.2cm}  p{0.2cm}  p{0.2cm} >{\raggedright\arraybackslash}p{4cm}}
		\toprule
%		\multirow{2}{*}{Paper} & \multirow{2}{*}{Objective}                                                                            & \multirow{2}{*}{Problem / Model} & \multirow{2}{*}{Solution / Method}             & \multirow{2}{*}{Type} & \multirow{2}{*}{Grid Constraints} & \multicolumn{5}{l}{Consideration of uncertainty in optimisation}                                    & \multirow{2}{*}{Remarkable Findings}                                                                                     \\ \cmidrule(lr){7-11}
%		&                                                                                                                &                                           &                                                         &                                &                                            & Renewables & Res. Demand & EV demand & EV availability & Prices &                                                                                                                                   \\ \midrule
\rot{\textbf{Paper}} & \textbf{Research Objective} & \textbf{Formulation} & \textbf{Method} & \rot{\textbf{V2G}} & \rot{\textbf{Grid Constr.}} & \rot{\textbf{Renewables}} & \rot{\textbf{Demand}} & \rot{\textbf{Battery}} & \rot{\textbf{Available}} & \rot{\textbf{Price}} & \textbf{Remarkable Findings} \\ \midrule
\endhead

\rowcolor[gray]{.95} \cite{OConnell2014}                    & Minimise charging costs                                                                                        & Receding horizon                          & Sequential quadratic programming                        & $\times$                 & \checkmark                                        & $\times$                  & \checkmark                & \checkmark              & \checkmark                    & \checkmark           & Effective reduction of forecast error impacts through rolling optimisation                       \\

\cite{Wang2017}                        & Minimise charging costs                                                                                        & Receding horizon                          & Model Predictive Control                                & $\times$                 & $\times$                                         & $\times$                  & $\times$                   & \checkmark                & \checkmark                      & $\times$              & Proved benefit of utilising receding horizon control and virtual load constraints                                                 \\

\rowcolor[gray]{.95} \cite{Vaya2015}                        & Minismise cost of charging by decentralised control                                                            & Receding horizon                          & Alternating direction method of multipliers             & $\times$                 & $\times$                                         & $\times$                  & $\times$                   & \checkmark                & \checkmark                      & $\times$              & Scalable parallel computation of local optimisation problems                                     \\

\cite{Soroudi2014}                    & Minimise total operating costs of energy procurement in distribution networks.                                 & Bi-level robust optimisation              & SNOPT Solver                                            & \checkmark                  & \checkmark                                        & $\times$                  & $\times$                   & $\times$                 & $\times$                       & \checkmark             & Immunisation of schedule to prediction errors to a controlled degree of risk-aversion. \\%

\rowcolor[gray]{.95} \cite{Bai2015}                         & Minimise the total cost of all thermal generators and all PEV aggregators                                      & Robust optimisation                       & Mixed-integer quadratic programming               & \checkmark                  & $\times$                                       & $\times$                  & $\times$                   & \checkmark                & \checkmark                      & \checkmark             & Aggregators participating in regulation services benefit the power grid and contain the uncertainty of costs of network operation \\

\cite{Korolko2015}                     & Minimise charging costs in an unregulated electricity market                                                   & Robust optimisation                       & Cutting plane method                                    & $\times$                 & $\times$                                         & $\times$                  & $\times$                   & $\times$                 & $\times$                       & \checkmark             & Feasible and robust solutions are achieved close to optimality with respect to uncertain market prices                            \\

\rowcolor[gray]{.95} \cite{Zhao2015}                        & Guarantee a predefined level of charging costs                                                                 & Stochastic optimisation                   & Information gap decision theory                         & \checkmark                  & $\times$                                         & $\times$                  & $\times$                   & $\times$                 & $\times$                       & \checkmark             & effective strategies to secure a predefined profit for risk-averse decision makers                                               \\

\cite{Zhou2017}                        & Minimise charging costs at a charging station with deadline constraints.                                       & Stochastic optimisation                   & Lyapunov optimisation                                   & $\times$                 & $\times$                                         & \checkmark                 & $\times$                   & \checkmark                & \checkmark                      & \checkmark             & robust algorithm points out that trade-off between cost and average fulfilment ratio of requsts is at most linear                 \\

\rowcolor[gray]{.95} \cite{Jin2014}                         & Minimise the cost of drawing non-renewable energy from the grid                                                & Stochastic optimisation                   & Lyapunov optimisation                                   & $\times$                 & $\times$                                         & \checkmark                 & $\times$                   & \checkmark                & $\times$                       & \checkmark             & Robust reduced charging costs and delay of charging                                                                         \\

 \cite{Zhang2014}                       & Minimise average length of charging under an average cost constraint                                           & Stochastic optimisation                   & Markov decision process (MDP)                           & $\times$                 & $\times$                                         & \checkmark                 & $\times$                   & $\times$                 & \checkmark                      & \checkmark             & -                                                                                                                               \\

\rowcolor[gray]{.95} \cite{Shi2011}                         & Profit maximisation for EV owner by participation on a regulation service market                               & Stochastic optimisation                   & Markov decision process (MDP) with Q-Learning algorithm & \checkmark                  & $\times$                                         & $\times$                  & $\times$                   & $\times$                 & $\times$                       & \checkmark             & Model price uncertainty via a Markov chain with unknown transition probabilities                                                  \\

\cite{Al-Awami2012}                    & Maximize a load-serving entity's profits through V2G services while maintaining risks within acceptable levels & Stochastic optimisation                   & CVaR on mixed-integer stochastic linear programme       & $\times$                 & $\times$                                         & $\times$                  & $\times$                   & $\times$                 & $\times$                       & \checkmark             & V2G services help to lower LSE risk levels and reduce emissions by displacing thermal generation \\

\rowcolor[gray]{.95}  \cite{Mehri2017}                       & Minimise charging cost and emissions of EV charging                                                            & Stochastic scenario tree                  & MINLP with epsilon-constraint method                    & \checkmark                  & \checkmark                                        & \checkmark                 & \checkmark                  & $\times$                 & $\times$                       & $\times$              & Reliable benefit for both system operator and EV owner resulted from bidirectional scheduling.  \\

\cite{Wu2017}                          & Scheduling in public charging stations with distributed generation to minimise charging costs                  & Two-stage stochastic optimisation         & Monte Carlo sample-average approximation technique      & \checkmark                 & $\times$                                      & \checkmark                 & \checkmark                  & \checkmark                & \checkmark                      & \checkmark             & Benefits in reducing charging costs, detrimental influence on the distribution network, and unsatisfied charging demand           \\
		
\rowcolor[gray]{.95} \cite{Liu2017}                         & Minimise charging costs                                                                                        & Stochastic linear programme               & Dynamic stochastic optimisation                         & $\times$                 & $\times$                                         & \checkmark                 & $\times$                & \checkmark                & \checkmark                      & \checkmark             & Achieved near-optimal results                                                 \\		 \bottomrule
\caption[Literature review of stochastic scheduling approaches]{Literature review of stochastic scheduling approaches and which uncertainties they consider (\checkmark = yes / $\times$ = no )}
\label{tab:unclr}
	\end{longtable}

\endgroup

\end{landscape}
%\restoregeometry



\section{Barriers and Challenges to EV Emergence and Scheduling}
\label{sec:barr}

For designing an optimal EV scheduling algorithm identifying potential facilitators and prevalent barriers for both EV market uptake and owners' engagement with aggregators is pivotal. Several obstacles must be overcome for widespread grid-friendly EV adoption, whereby social and technical barriers are of equal importance. For an overview of challenges confer \Autoref{tab:barriers}.

According to a survey, the biggest social concerns about electric vehicles stem from high cost, limited battery range, battery degradation, reliability and charging infrastructure \cite{Egbue2012}. Especially, the high upfront investment cost provides a disincentive to acquiring electric vehicles, while range anxiety of users of electric vehicles is increasingly solved by the advent of modern Lithium-ion batteries \cite{Tan2016a}. The need to estimate the present value of fuel cost savings and devised charging control requires effort and yields indefinite results as estimates are based on a volatile market \cite{Sovacool2009}. It was moreover shown that compared to other domestic energy efficiency measures, electric vehicles exhibit long payback periods \cite{Sovacool2009}. But also non-financial factors such as doubt about the sustainability of electrified transport, risk-aversion and security concerns about an unproven technology and the general public perception play a vital role in shaping purchase decisions \cite{Egbue2012}.

When additionally taking remote charging control into account, further issues arise. A prerequisite of demand side management (DSM) is suitable local communication infrastructure to control distributed flexible loads, which is not widely available to date \cite{Budka2014}. Even if ICT infrastructure is provided, increasingly complex grid control (due to many tiny controllable units) and energy losses (due to storage and potential feed back to the grid) constitute a major challenge \cite{Strbac2008, Dehaghani2012}. Most importantly, the limits of usable load flexibility need to be addressed. Because the supply of ancillary services is not the primary purpose of electric vehicles, flexibility is constrained by reliably fulfilling the users' mobility demands \cite{Dehaghani2012}. 

% Please add the following required packages to your document preamble:
% \usepackage{booktabs}
\begin{table}[]
	\centering
	%\footnotesize
	\resizebox{\textwidth}{!} {%
		\begin{tabular}{@{}lll@{}}
			\toprule
			\textbf{Social barriers}                     & \textbf{Socio-technical barriers} & \textbf{Technical barriers}              \\ \midrule
			\tabitem high investment costs               & \tabitem driving range            & \tabitem lack of ICT infrastructure      \\
			\tabitem risk-aversion / security concerns & \tabitem battery degradation      & \tabitem complexity of network operation \\
			\tabitem privacy preservation                & \tabitem reliability              & \tabitem energy conversion losses (V2G)  \\
			\tabitem credibility of sustainability       & \tabitem regulatory framework     &                                          \\
			\tabitem sustainability awareness            &                                   &                                          \\
			\tabitem leniency and lack of simplicity     &                                   &                                          \\
			\tabitem public perception                   &                                   &                                          \\ \bottomrule
		\end{tabular}
	}
	\caption{Barriers and challenges to EV emergence and scheduling}
	\label{tab:barriers}
\end{table}

Consequently, the social question arises to what extent consumers are willing to contribute to the integration of electric vehicles plus renewable energy. Besides reliability issues, the utilisation of EV batteries for DSM is proven to accelerate battery degradation due to more frequent charge/discharge cycles and limit the usable battery capacity \cite{Peterson2010, Wang2013a}. Consequently, batteries must be replaced more frequently and participants must be reimbursed to make DSM attractive \cite{Dehaghani2012}. Reference \cite{Dutschke2013} conducted a survey asking participants for their preference regarding different tariff-schemes and charging control modes. While the largest group of participants favoured an extensive V2G-tariff with direct load control and bidirectional power flow yielding the greatest economic benefit, a similar amount of respondents preferred to adhere to single-rate tariffs. When asked about what would alter their decision to participate in devised charging schemes common factors were monetary compensation, the level of automation, and simplicity \cite{Dutschke2013}. A different study further highlighted concerns about privacy preservation due to the high degree of information exchange \cite{Sovacool2009}.

The studies underline that not sustainability but \textit{cost} and \textit{reliability} shape decisions to purchase EVs and engage in remote scheduling schemes. Thus, an economic not intrinsic drive to support the grid integration of renewables preponderates. In consequence, for a quick decarbonisation of the transport sector and customer acceptance of devised charging control, an aggregator should prioritise cost saving by wholesale market participation and demand satisfaction objectives under technical constraints, which optionally avoid control sequences detrimental to battery life.

%%% REGULATORY CONCERNS
%- "development of the regulatory framework needs to support more decentralized decisions and system operation, and must potentially be adapted to respond to structural changes imposed by the large numbers of distributed energy resources and adaptive loads like EVs" \cite{Schuller2013}
% - solutions: strong warranties, battery swap programs, increased tax credits, charging infrastrucutre investments
%- inappropriate market structure / lack of incentives